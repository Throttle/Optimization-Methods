\chapter{Описание задачи}
\label{cha:analysis}

Функция для минимизации (\eqref{F:function}):

\begin{equation}
y=\sinh(\frac{3x^4 - x + \sqrt{17}-3}{2}) + \sin{\frac{5^\frac{1}{3}x^3-5^\frac{1}{3}x + 1 - 2\cdot 5^\frac{1}{3}}{-x^3+x+2}}.
\label{F:function}
\end{equation}

Отрезок поиска: $x\in [0,1]$.

\section{Лабораторная работа \No1}
В таблице \ref{tb:tab1} приведены результаты работы \b{метода поразрядного поиска}.

\begin{table}[!ht]
\caption{Результаты работы метода}
\begin{tabular}{|p{0.03\textwidth}|p{0.22\textwidth}|p{0.37\textwidth}|p{0.11\textwidth}|p{0.13\textwidth}|}
\hline
\No & Заданная точность & Количество вычислений функции & $x*$ & $f(x*)$\\
\hline
1 & 0.01 & 15 & 0.453125 & -0.550957 \\
\hline
2 & 0.0001 & 30 & 0.442139 & -0.551190 \\
\hline
3 & 0.000001 & 45 & 0.442368 & -0.551190 \\
\hline
\end{tabular}
\label{tb:tab1}
\end{table}

\section{Лабораторная работа \No2}
В таблице \ref{tb:tab2} приведены результаты работы \b{метода золотого сечения}.

\begin{table}[!ht]
\caption{Результаты работы метода}
\begin{tabular}{|p{0.03\textwidth}|p{0.22\textwidth}|p{0.37\textwidth}|p{0.11\textwidth}|p{0.13\textwidth}|}
\hline
\No & Заданная точность & Количество вычислений функции & $x*$ & $f(x*)$\\
\hline
1 & 0.01 & 12 & 0.442719 & -0.551187 \\
\hline
2 & 0.0001 & 22 & 0.442357 & -0.551190 \\
\hline
3 & 0.000001 & 31 & 0.442364 & -0.551190 \\
\hline
\end{tabular}
\label{tb:tab2}
\end{table}

\section{Лабораторная работа \No3}
В таблице \ref{tb:tab3} приведены результаты работы \b{метода квадратичной интерполяции в сочетании с методом золотого сечения}. Количество итераций метода золотого сечения --- 4.

\begin{table}[!ht]
\caption{Результаты работы метода}
\begin{tabular}{|p{0.03\textwidth}|p{0.22\textwidth}|p{0.37\textwidth}|p{0.11\textwidth}|p{0.13\textwidth}|}
\hline
\No & Заданная точность & Количество вычислений функции & $x*$ & $f(x*)$\\
\hline
1 & 0.01 & 13 & 0.441913 & -0.551190 \\
\hline
2 & 0.0001 & 19 & 0.442360 & -0.551190 \\
\hline
3 & 0.000001 & 25 & 0.442364 & -0.551190 \\
\hline
\end{tabular}
\label{tb:tab3}
\end{table}

\section{Лабораторная работа \No4}
В таблице \ref{tb:tab4} приведены результаты работы \b{модифицированного метода Ньютона}.

\begin{table}[!ht]
\caption{Результаты работы метода}
\begin{tabular}{|p{0.03\textwidth}|p{0.22\textwidth}|p{0.37\textwidth}|p{0.11\textwidth}|p{0.13\textwidth}|}
\hline
\No & Заданная точность & Количество вычислений функции & $x*$ & $f(x*)$\\
\hline
1 & 0.01 & 12 & 0.442370 & -0.551185 \\
\hline
2 & 0.0001 & 15 & 0.442364 & -0.551190 \\
\hline
3 & 0.000001 & 18 & 0.442364 & -0.551190 \\
\hline
\end{tabular}
\label{tb:tab4}
\end{table}



\section{Сводная таблица}
В таблице представлены результаты работы рассмотренных методов для точности $0.000001$.

\begin{table}[!ht]
\caption{Сводная таблица результатов работы методов}
\begin{tabular}{|p{0.03\textwidth}|p{0.39\textwidth}|p{0.20\textwidth}|p{0.11\textwidth}|p{0.13\textwidth}|}
\hline
\No & Метод & Количество вычислений функции & $x*$ & $f(x*)$\\
\hline
1 & поразрядного поиска & 45 & 0.442368 & -0.551190 \\
\hline
2 & золотого сечения & 31 & 0.442364 & -0.551190 \\
\hline
3 & квадратичной интерполяции в сочетании с методом золотого сечения &  25 & 0.442364 & -0.551190 \\
\hline
4 & модифицированный метод Ньютона & 18 & 0.442364 & -0.551190 \\
\hline
5 & fminbnd & 12 & 0.442364 & -0.551190 \\
\hline
\end{tabular}
\label{tb:tab4}
\end{table}

%
% % В начале раздела  можно напомнить его цель
%

%%% Local Variables: 
%%% mode: latex
%%% TeX-master: "rpz"
%%% End: 
