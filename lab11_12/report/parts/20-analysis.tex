\chapter{Описание работы}
\section{Задание}
Для квадратичной функции $f$ задать три ограничения (два линейных и одно нелинейное) в
виде нестрогих неравенств, чтобы:

\begin{enumerate}
	\item допустимое множество было выпуклым;
	\item точка минимума квадратичной функции не принадлежала допустимому множеству.
\end{enumerate}

Функция для минимизации (\eqref{F:function}):

\begin{equation}
y=4 x_{1} x_{2} + 7x_1^2 + 4x_2^2 + 6 \sqrt{5}x_1 -12\sqrt{5}x_2 + 51.
\label{F:function}
\end{equation}

\section{Решение}

Стартовая точка: $x\in [0,-\sqrt{5}]$.

\subsection{Выбранные ограничения}
Выбранные ограничения(\eqref{F:borders}): 

\begin{equation}
\left\{\begin{matrix}
-2x_1+3x_2-14
\\ -x_1-3
\\ (x_1+2)^2+(x_2-4)^2-15
\end{matrix}\right.
\label{F:borders}
\end{equation}

\subsection{Минимизация квадратичной функции $f$}
В таблице \ref{tb:tab1} приведены результаты работы.

\begin{table}[!ht]
\caption{Результаты работы метода}
\begin{tabular}{|p{0.15\textwidth}|p{0.15\textwidth}|p{0.15\textwidth}|p{0.25\textwidth}|p{0.13\textwidth}|}
\hline
Метод & Точность& Количество вычислений функции & $X$ & $f(X)$\\
\hline
штрафных функций & 0.001 & 993 & [-1.710983, 3.532580] & -20.512324\\
\hline
барьерных функций & 0.001 & 1304 & [-1.625131, 3.583246] & -20.399014\\
\hline
\end{tabular}
\label{tb:tab1}
\end{table}


%
% % В начале раздела  можно напомнить его цель
%

%%% Local Variables: 
%%% mode: latex
%%% TeX-master: "rpz"
%%% End: 
