\chapter{Описание задачи}
\label{cha:analysis}
\section{Часть 1}
Функция для минимизации (\eqref{F:function}):

\begin{equation}
y=4 x_{1} x_{2} + 7x_1^2 + 4x_2^2 + 6 \sqrt{5}x_1 -12\sqrt{5}x_2 + 51.
\label{F:function}
\end{equation}

Базовая точка: $x\in [0,-\sqrt{5}]$.

\section{Поиск точки минимума по теоритической формуле}
Минимум функции достигается при условии:
\[
\left\{ 
\begin{array}{l}
  \dfrac{\partial f_1}{\partial x_1} = 0 \\
  \dfrac{\partial f_1}{\partial x_2} = 0 \\
  \dfrac{\partial^2 f_1}{\partial x_1^2} > 0 \\
  \dfrac{\partial^2 f_1}{\partial x_2^2} > 0
\end{array} \right.
\]

\[
\left\{ 
\begin{array}{l l l}
	4 x_2 + 14 x_1 + 6 \sqrt{5} = 0 \\
	4 x_1 + 8 x_2 + 6 \sqrt{5} x_1 - 12 \sqrt{12} = 0 \\
	14 > 0 \\
	8 > 0
\end{array} \right.
\]

Таким образом:
$X_{min}^T = \left( -\sqrt{5}, - 2 \sqrt{5}\right) \approx (-2.2261 ; -4.4721), f_{min}(X_{min}^T) = -24.00$

\subsection{Результаты расчетов}
В таблице \ref{tb:tab1} приведены результаты работы \b{метода минимизации по правильному симплексу}. Первоначальная длина ребра симплекса: $a=0.5$.

\begin{table}[!ht]
\caption{Результаты работы метода}
\begin{tabular}{|p{0.15\textwidth}|p{0.15\textwidth}|p{0.15\textwidth}|p{0.25\textwidth}|p{0.13\textwidth}|}
\hline
Метод & Точность& Количество вычислений функции & $X$ & $f(X)$\\
\hline
метод & 0.01 & 161 & [-2.236068, 4.472136] & -24.000000 \\
сопряженных & 0.0001 & 161 & [-2.236068, 4.472136] & -24.000000\\
градиентов & 0.000001 & 161 & [-2.236068, 4.472136]  & -24.000000 \\
\hline
метод & 0.01 & 21 & [-2.236068, 4.472136] & -24.000000 \\
Ньютона & 0.0001 & 21 & [-2.236068, 4.472136] & -24.000000\\
 & 0.000001 & 21 & [-2.236068, 4.472136] & -24.000000\\
\hline
& 0.01 & 164 & [-2.236068, 4.472136] & -24.000000\\
ДФП& 0.0001 & 164 & [-2.236068, 4.472136] & -24.000000\\
& 0.000001 & 164 & [-2.236068, 4.472136] & -24.000000\\
\hline
& 0.01 & 101 & [-2.237092, 4.475051] & -23.999971 \\
fminsearch& 0.0001 & 127 & [-2.236047, 4.472151] & -24.000000\\
& 0.000001 & 151 & [-2.236047, 4.472151] & -24.000000 \\
\hline
\end{tabular}
\label{tb:tab1}
\end{table}

\section{Часть 2}

В таблице \ref{tb:tab23} представлены результаты работы рассмотренных методов для точности $0.000001$.

\begin{table}[!ht]
\caption{Результаты работы метода}
\begin{tabular}{|p{0.15\textwidth}|p{0.15\textwidth}|p{0.15\textwidth}|p{0.25\textwidth}|p{0.13\textwidth}|}
\hline
Метод & Точность& Количество вычислений функции & $X$ & $f(X)$\\
\hline
метод & 0.01 & 79 & [2.487193, 0.157994] & 2.385103 \\
сопряженных & 0.0001 & 191 & [2.494021, 0.156760] & 2.385086\\
градиентов & 0.000001 & 2065 & [9.520215, 0.041673] & 2.381174 \\
\hline
метод & 0.01 & 94 & [8.994507, 0.044107] & 2.381187\\
Ньютона & 0.0001 & 107 & [9.109737, 0.043562] & 2.381184\\
 & 0.000001 & 328 & [10.501650, 0.037791] & 2.381156 \\
\hline
& 0.01 & 81 & [2.487193, 0.157994] & 2.385103\\
ДФП& 0.0001 & 81 & [2.487193, 0.157994] & 2.385103\\
& 0.000001 & 81 & [2.487193, 0.157994] & 2.385103 \\
\hline
& 0.01 & 401 & [21.717703, 0.018274] & 2.381108 \\
fminsearch& 0.0001 & 401 & [21.717703, 0.018274] & 2.381108 \\
& 0.000001 & 401 & [21.717703, 0.018274] & 2.381108 \\
\hline
\end{tabular}
\label{tb:tab23}
\end{table}


%
% % В начале раздела  можно напомнить его цель
%

%%% Local Variables: 
%%% mode: latex
%%% TeX-master: "rpz"
%%% End: 
