\sloppy 

% 1. Настройки стиля ГОСТ 7-32
% Для начала определяем, хотим мы или нет, чтобы рисунки и таблицы нумеровались в пределах раздела, или нам нужна сквозная нумерация.
% А не забыл ли автор букву 't' ?
\EqInChapter % формулы будут нумероваться в пределах раздела
\TableInChapter % таблицы будут нумероваться в пределах раздела
\PicInChapter % рисунки будут нумероваться в пределах раздела



% 2. Добавляем гипертекстовое оглавление в PDF
\usepackage[
bookmarks=true, colorlinks=true, unicode=true,
urlcolor=black,linkcolor=black, anchorcolor=black,
citecolor=black, menucolor=black, filecolor=bla,
]{hyperref}

\usepackage{pscyr}
\renewcommand{\rmdefault}{fjn} % скажем "нет" таймсу и компьютер модерну (даже супер-варианту)

\usepackage{totcount}
\regtotcounter{page}
\regtotcounter{figure} % не работает :(
\regtotcounter{table} % не работает :(

\newtotcounter{myfigure}


% 3. Изменение начертания шрифта --- после чего выглядит таймсоподобно.
% apt-get install scalable-cyrfonts-tex

\IfFileExists{cyrtimes.sty}
    {
        \usepackage{cyrtimespatched}
    } 
    {
        % А если Times нету, то будет CM...
    }

% 4. Прочие полезные пакеты.
\usepackage{underscore} % Ура! Теперь можно писать подчёркивание.
\usepackage{graphicx}   % Пакет для включения рисунков
 
 
 % 5. Любимые команды
\newcommand{\Code}[1]{\textbf{#1}}

\renewcommand{\i}[1]{\emph{#1}}
\renewcommand{\b}[1]{\textbf{#1}}
\renewcommand{\`}[1]{\texttt{#1}}

\newcommand{\ol}{\begin{enumerate}}
\newcommand{\lo}{\end{enumerate}}
\newcommand{\ul}{\begin{itemize}}
\newcommand{\lu}{\end{itemize}}
\newcommand{\li}{\item}

\usepackage{verbatim}       % не
\usepackage{listingsutf8}     % катит
\lstset{
    inputencoding=utf8x,
    extendedchars=\true% чтобы в листингах не убирался первый символ из начала русского слова
    } 
\lstloadlanguages{Python,SQL}%Загружаемые языки
%\usepackage{biblatex}

\usepackage{fancyvrb}  % катит (еще бы word-wrap туда)

% 6. Поля
% С такими оно полями оно работает по-умолчанию:
% \RequirePackage[left=20mm,right=10mm,top=20mm,bottom=20mm,headsep=0pt]{geometry}
% Если вас тошнит от поля в 10мм --- увеличивайте до 20-ти, ну и про переплёт не забывайте:
\geometry{right=20mm}
\geometry{left=30mm}
