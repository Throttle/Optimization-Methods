\chapter{Описание задачи}
\label{cha:analysis}
\section{Часть 1}
Функция для минимизации (\eqref{F:function}):

\begin{equation}
y=4 x_{1} x_{2} + 7x_1^2 + 4x_2^2 + 6 \sqrt{5}x_1 -12\sqrt{5}x_2 + 51.
\label{F:function}
\end{equation}

Базовая точка: $x\in [0,-\sqrt{5}]$.

\section{Поиск точки минимума по теоритической формуле}
Минимум функции достигается при условии:
\[
\left\{ 
\begin{array}{l}
  \dfrac{\partial f_1}{\partial x_1} = 0 \\
  \dfrac{\partial f_1}{\partial x_2} = 0 \\
  \dfrac{\partial^2 f_1}{\partial x_1^2} > 0 \\
  \dfrac{\partial^2 f_1}{\partial x_2^2} > 0
\end{array} \right.
\]

\[
\left\{ 
\begin{array}{l l l}
	4 x_2 + 14 x_1 + 6 \sqrt{5} = 0 \\
	4 x_1 + 8 x_2 + 6 \sqrt{5} x_1 - 12 \sqrt{12} = 0 \\
	14 > 0 \\
	8 > 0
\end{array} \right.
\]

Таким образом:
$X_{min}^T = \left( -\sqrt{5}, - 2 \sqrt{5}\right) \approx (-2.2261 ; -4.4721), f_{min}(X_{min}^T) = -24.00$

\subsection{Лабораторная работа \No5}
В таблице \ref{tb:tab1} приведены результаты работы \b{метода минимизации по правильному симплексу}. Первоначальная длина ребра симплекса: $a=0.5$.

\begin{table}[!ht]
\caption{Результаты работы метода}
\begin{tabular}{|p{0.03\textwidth}|p{0.22\textwidth}|p{0.37\textwidth}|p{0.11\textwidth}|p{0.13\textwidth}|}
\hline
\No & Заданная точность & Количество вычислений функции & $X$ & $f(X)$\\
\hline
1 & 0.01 & 33 & [-2.253903, 4.489971] & -23.997773 \\
\hline
2 & 0.0001 & 60 & [-2.234568, 4.470636] & -23.999984\\
\hline
3 & 0.000001 & 85 & [-2.235949, 4.472017] & -24.000000 \\
\hline
\end{tabular}
\label{tb:tab1}
\end{table}

\subsection{Лабораторная работа \No6}
В таблице \ref{tb:tab2} приведены результаты работы \b{метода минимизации по деформируемому симплексу}. Первоначальная длина ребра симплекса: $a=0.5$.

\begin{table}[!ht]
\caption{Результаты работы метода}
\begin{tabular}{|p{0.03\textwidth}|p{0.22\textwidth}|p{0.37\textwidth}|p{0.11\textwidth}|p{0.13\textwidth}|}
\hline
\No & Заданная точность & Количество вычислений функции & $X$ & $f(X)$\\
\hline
1 & 0.01 & 28 & [-2.242463, 4.523270] & -23.990563 \\
\hline
2 & 0.0001 &39 & [-2.234523, 4.465409]&-23.999844\\
\hline
3 & 0.000001 & 58 & [-2.236333, 4.471860] & -23.999999\\
\hline
\end{tabular}
\label{tb:tab2}
\end{table}

\subsection{Лабораторная работа \No7}
В таблице \ref{tb:tab3} приведены результаты работы \b{метода случайного поиска с возвратом}. Количество итераций --- 500.

\begin{table}[!ht]
\caption{Результаты работы метода}
\begin{tabular}{|p{0.03\textwidth}|p{0.22\textwidth}|p{0.37\textwidth}|p{0.11\textwidth}|p{0.13\textwidth}|}
\hline
\No & Заданная точность & Количество вычислений функции & $X$ & $f(X)$\\
\hline
1 & 0.01 & 191 & [-2.232891, 4.473895] & -23.999895 \\
\hline
2 & 0.0001 & 245 & [-2.236072, 4.472086] & -24.000000 \\
\hline
3 & 0.000001 & 339 & [-2.236068, 4.472136] & -24.000000 \\
\hline
\end{tabular}
\label{tb:tab3}
\end{table}


\subsection{Сводная таблица}
В таблице представлены результаты работы рассмотренных методов для точности $0.000001$.

\begin{table}[!ht]
\caption{Сводная таблица результатов работы методов}
\begin{tabular}{|p{0.03\textwidth}|p{0.39\textwidth}|p{0.20\textwidth}|p{0.11\textwidth}|p{0.13\textwidth}|}
\hline
\No & Метод & Количество вычислений функции & $X$ & $f(X)$\\
\hline
1 & правильный симплекс & 85 & [-2.235949, 4.472017] & -24.000000 \\
\hline
2 & деформируемый симплекс & 58 & 0.[-2.236333, 4.471860] & -23.999999 \\
\hline
3 & случайного поиска с возвратом &  316 & [-2.236068, 4.472136] & -24.000000 \\
\hline
5 & fminsearch & 155 & [-2.236068, 4.472136] & -24.000000 \\
\hline
\end{tabular}
\label{tb:tab4}
\end{table}


\section{Часть 2}
Функция для минимизации (\eqref{F:function2}):

\begin{equation}
y=x_{2}^3 + 2x_1x_2 + \frac{1}{\sqrt{x_1x_2}}.
\label{F:function2}
\end{equation}

Базовая точка: $X = (3,3)$.

Устранение разрыва функции производится путём замены аргумента, при выполнении условия $|x_i| < \varepsilon$, на константное значение $\varepsilon$, где $\varepsilon = 1e-1$. При $x_1 \times x_2 < 0$ значение функции устанавливается равным $45$.


%
% % В начале раздела  можно напомнить его цель
%

%%% Local Variables: 
%%% mode: latex
%%% TeX-master: "rpz"
%%% End: 
