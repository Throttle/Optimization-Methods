%% Преамбула TeX-файла

% 1. Стиль и язык
\documentclass[10pt]{style/G7-32} % Стиль (по умолчанию будет 14pt)

% Остальные стандартные настройки убраны в preamble-std.tex
\sloppy 

% 1. Настройки стиля ГОСТ 7-32
% Для начала определяем, хотим мы или нет, чтобы рисунки и таблицы нумеровались в пределах раздела, или нам нужна сквозная нумерация.
% А не забыл ли автор букву 't' ?
\EqInChapter % формулы будут нумероваться в пределах раздела
\TableInChapter % таблицы будут нумероваться в пределах раздела
\PicInChapter % рисунки будут нумероваться в пределах раздела



% 2. Добавляем гипертекстовое оглавление в PDF
\usepackage[
bookmarks=true, colorlinks=true, unicode=true,
urlcolor=black,linkcolor=black, anchorcolor=black,
citecolor=black, menucolor=black, filecolor=bla,
]{hyperref}

\usepackage{pscyr}
\renewcommand{\rmdefault}{fjn} % скажем "нет" таймсу и компьютер модерну (даже супер-варианту)

\usepackage{totcount}
\regtotcounter{page}
\regtotcounter{figure} % не работает :(
\regtotcounter{table} % не работает :(

\newtotcounter{myfigure}


% 3. Изменение начертания шрифта --- после чего выглядит таймсоподобно.
% apt-get install scalable-cyrfonts-tex

\IfFileExists{cyrtimes.sty}
    {
        \usepackage{cyrtimespatched}
    } 
    {
        % А если Times нету, то будет CM...
    }

% 4. Прочие полезные пакеты.
\usepackage{underscore} % Ура! Теперь можно писать подчёркивание.
\usepackage{graphicx}   % Пакет для включения рисунков
 
 
 % 5. Любимые команды
\newcommand{\Code}[1]{\textbf{#1}}

\renewcommand{\i}[1]{\emph{#1}}
\renewcommand{\b}[1]{\textbf{#1}}
\renewcommand{\`}[1]{\texttt{#1}}

\newcommand{\ol}{\begin{enumerate}}
\newcommand{\lo}{\end{enumerate}}
\newcommand{\ul}{\begin{itemize}}
\newcommand{\lu}{\end{itemize}}
\newcommand{\li}{\item}

\usepackage{verbatim}       % не
\usepackage{listingsutf8}     % катит
\lstset{
    inputencoding=utf8x,
    extendedchars=\true% чтобы в листингах не убирался первый символ из начала русского слова
    } 
\lstloadlanguages{Python,SQL}%Загружаемые языки
%\usepackage{biblatex}

\usepackage{fancyvrb}  % катит (еще бы word-wrap туда)

% 6. Поля
% С такими оно полями оно работает по-умолчанию:
% \RequirePackage[left=20mm,right=10mm,top=20mm,bottom=20mm,headsep=0pt]{geometry}
% Если вас тошнит от поля в 10мм --- увеличивайте до 20-ти, ну и про переплёт не забывайте:
\geometry{right=20mm}
\geometry{left=30mm}

\usepackage{multirow}
\usepackage{lscape}
\makeatletter
\newcommand{\rmnum}[1]{\romannumeral #1}
\newcommand{\Rmnum}[1]{\expandafter\@slowromancap\romannumeral #1@}
\makeatother
\begin{document}

\frontmatter % выключает нумерацию ВСЕГО; здесь начинаются ненумерованные главы: реферат, введение, глоссарий, сокращения и прочее

%\include{parts/00-abstract}

%\tableofcontents

%\include{parts/10-defines}
%\include{parts/11-abbrev}

%\include{parts/12-intro}

\mainmatter % это включает нумерацию глав и секций в документе ниже

\chapter{Описание задачи}
\label{cha:analysis}
\section{Часть 1}
Функция для минимизации (\eqref{F:function}):

\begin{equation}
y=4 x_{1} x_{2} + 7x_1^2 + 4x_2^2 + 6 \sqrt{5}x_1 -12\sqrt{5}x_2 + 51.
\label{F:function}
\end{equation}

Базовая точка: $x\in [0,-\sqrt{5}]$.

\section{Поиск точки минимума по теоритической формуле}
Минимум функции достигается при условии:
\[
\left\{ 
\begin{array}{l}
  \dfrac{\partial f_1}{\partial x_1} = 0 \\
  \dfrac{\partial f_1}{\partial x_2} = 0 \\
  \dfrac{\partial^2 f_1}{\partial x_1^2} > 0 \\
  \dfrac{\partial^2 f_1}{\partial x_2^2} > 0
\end{array} \right.
\]

\[
\left\{ 
\begin{array}{l l l}
	4 x_2 + 14 x_1 + 6 \sqrt{5} = 0 \\
	4 x_1 + 8 x_2 + 6 \sqrt{5} x_1 - 12 \sqrt{12} = 0 \\
	14 > 0 \\
	8 > 0
\end{array} \right.
\]

Таким образом:
$X_{min}^T = \left( -\sqrt{5}, - 2 \sqrt{5}\right) \approx (-2.2261 ; -4.4721), f_{min}(X_{min}^T) = -24.00$

\subsection{Лабораторная работа \No5}
В таблице \ref{tb:tab1} приведены результаты работы \b{метода минимизации по правильному симплексу}. Первоначальная длина ребра симплекса: $a=0.5$.

\begin{table}[!ht]
\caption{Результаты работы метода}
\begin{tabular}{|p{0.03\textwidth}|p{0.22\textwidth}|p{0.37\textwidth}|p{0.11\textwidth}|p{0.13\textwidth}|}
\hline
\No & Заданная точность & Количество вычислений функции & $X$ & $f(X)$\\
\hline
1 & 0.01 & 33 & [-2.253903, 4.489971] & -23.997773 \\
\hline
2 & 0.0001 & 60 & [-2.234568, 4.470636] & -23.999984\\
\hline
3 & 0.000001 & 85 & [-2.235949, 4.472017] & -24.000000 \\
\hline
\end{tabular}
\label{tb:tab1}
\end{table}

\subsection{Лабораторная работа \No6}
В таблице \ref{tb:tab2} приведены результаты работы \b{метода минимизации по деформируемому симплексу}. Первоначальная длина ребра симплекса: $a=0.5$.

\begin{table}[!ht]
\caption{Результаты работы метода}
\begin{tabular}{|p{0.03\textwidth}|p{0.22\textwidth}|p{0.37\textwidth}|p{0.11\textwidth}|p{0.13\textwidth}|}
\hline
\No & Заданная точность & Количество вычислений функции & $X$ & $f(X)$\\
\hline
1 & 0.01 & 28 & [-2.242463, 4.523270] & -23.990563 \\
\hline
2 & 0.0001 &39 & [-2.234523, 4.465409]&-23.999844\\
\hline
3 & 0.000001 & 58 & [-2.236333, 4.471860] & -23.999999\\
\hline
\end{tabular}
\label{tb:tab2}
\end{table}

\subsection{Лабораторная работа \No7}
В таблице \ref{tb:tab3} приведены результаты работы \b{метода случайного поиска с возвратом}. Количество итераций --- 500.

\begin{table}[!ht]
\caption{Результаты работы метода}
\begin{tabular}{|p{0.03\textwidth}|p{0.22\textwidth}|p{0.37\textwidth}|p{0.11\textwidth}|p{0.13\textwidth}|}
\hline
\No & Заданная точность & Количество вычислений функции & $X$ & $f(X)$\\
\hline
1 & 0.01 & 191 & [-2.232891, 4.473895] & -23.999895 \\
\hline
2 & 0.0001 & 245 & [-2.236072, 4.472086] & -24.000000 \\
\hline
3 & 0.000001 & 339 & [-2.236068, 4.472136] & -24.000000 \\
\hline
\end{tabular}
\label{tb:tab3}
\end{table}


\subsection{Сводная таблица}
В таблице представлены результаты работы рассмотренных методов для точности $0.000001$.

\begin{table}[!ht]
\caption{Сводная таблица результатов работы методов}
\begin{tabular}{|p{0.03\textwidth}|p{0.39\textwidth}|p{0.20\textwidth}|p{0.11\textwidth}|p{0.13\textwidth}|}
\hline
\No & Метод & Количество вычислений функции & $X$ & $f(X)$\\
\hline
1 & правильный симплекс & 85 & [-2.235949, 4.472017] & -24.000000 \\
\hline
2 & деформируемый симплекс & 58 & 0.[-2.236333, 4.471860] & -23.999999 \\
\hline
3 & случайного поиска с возвратом &  316 & [-2.236068, 4.472136] & -24.000000 \\
\hline
5 & fminsearch & 155 & [-2.236068, 4.472136] & -24.000000 \\
\hline
\end{tabular}
\label{tb:tab4}
\end{table}


\section{Часть 2}
Функция для минимизации (\eqref{F:function2}):

\begin{equation}
y=x_{2}^3 + 2x_1x_2 + \frac{1}{\sqrt{x_1x_2}}.
\label{F:function2}
\end{equation}

Базовая точка: $X = (3,3)$.

Устранение разрыва функции производится путём замены аргумента, при выполнении условия $|x_i| < \varepsilon$, на константное значение $\varepsilon$, где $\varepsilon = 1e-1$. При $x_1 \times x_2 < 0$ значение функции устанавливается равным $45$.


%
% % В начале раздела  можно напомнить его цель
%

%%% Local Variables: 
%%% mode: latex
%%% TeX-master: "rpz"
%%% End: 

%\include{parts/40-design}
%\include{parts/50-impl}
%\include{parts/30-research}
%\include{parts/60-economy}
%\include{parts/70-ecology}


\backmatter %% Здесь заканчивается нумерованная часть документа и начинаются ссылки и заключение

%\include{parts/80-conclusion}

%\include{parts/81-biblio}

\appendix   % Тут идут приложения

%\include{parts/90-appendix1}
%\include{parts/91-appendix2}
%\include{parts/92-appendix3}
%\include{parts/93-appendix4}

\end{document}

%%% Local Variables: 
%%% mode: latex
%%% TeX-master: t
%%% End: 
